\documentclass[10pt,a4paper]{article}
%\usepackage{endfloat}
\usepackage{f1000_styles}
\usepackage{units}
\usepackage[colorlinks]{hyperref}
\usepackage{url}
\usepackage{booktabs}
\usepackage[sort,compress]{natbib}
\usepackage{multicol}

\begin{document}

\newcommand{\NShots}{870}


\title{An annotation of cuts, depicted locations, and temporal progression in
the motion picture ``Forrest Gump''}

\author[1]{Christian Häusler}
\author[1,2]{Michael Hanke}

\affil[1]{Psychoinformatics Lab, Department of Psychology, University of Magdeburg, Universit\"{a}tsplatz 2, 39106 Magdeburg, Germany}
\affil[2]{Center for Behavioral Brain Sciences, Magdeburg, Germany}
\maketitle
\thispagestyle{fancy}

\begin{multicols}{1}
\begin{abstract}
% Abstracts should be up to 300 words and provide a succinct summary of the
% article. Although the abstract should explain why the article might be
% interesting, care should be taken not to inappropriately over-emphasise the
% importance of the work described in the article. Citations should not be used
% in the abstract, and the use of abbreviations should be minimized.

% 118 words
Here we present an annotation of locations and temporal progression depicted in
the movie ``Forrest Gump'', as an addition to a large public functional brain
imaging dataset (\url{http://studyforrest.org}). The annotation provides
information about the exact timing of each of the \NShots\ shots, and the
depicted location after every cut with a high, medium, and low level of
abstraction. Additionally, four classes are used to distinguish the differences
of the depicted time between shots. Each shot is also annotated regarding the
type of location (interior/exterior) and time of day. This annotation enables
further studies of visual perception, memory of locations, and the perception of
time under conditions of real-life complexity using the studyforrest dataset.

\end{abstract}
\end{multicols}

\begin{multicols}{2}
%The format of the main body of the article is flexible: it should be concise
%and in the format most appropriate to displaying the content of the article.

\section*{Introduction}
% ~380 words
% Rationale for creating the dataset(s) and/or objectives for the experiment
% resulting in the dataset - why the data were gathered or produced.

Cognitive neuroimaging research is moving towards studying brain behavior under
conditions of real-life-like complexity, and motion pictures are being utilized
with increasing frequency as stimuli in ``neurocinematics'' studies
\citep{hasson_2008_neurocinematics}. What sets motion pictures apart from other
dynamic naturalistic stimuli is that they are more likely to evoke time-locked
response patterns in a larger portion of the brain while retaining synchrony
across multiple individuals who are experiencing the same movie
\citep{hasson_2009_natural_stim_review,lankinen_2014_MEG_during_movie}. One
likely reason for this is the structure of movies. They are typically not
prolonged, contiguous captures of an environment from a first person
perspective, but rather they are carefully assembled, using ``cuts'', from
hundreds of short sequences shot from a variety of perspectives
\citep{cutting_2011_changing_poetics_of_dissolve}. These cuts are sharp
discontinuities in the sensory input that require all viewers to re-assess the
depicted environment in order to perform a cognitive re-orientation in fictional
space and time. This re-orientation can be complex and involve a
large bandwidth of cognitive processes: detection of familiar settings,
retrieval of prior knowledge from memory, discovery of change in locales and
depicted characters. Consequently movies, and their cuts in particular, offer an
excellent instrument to study complex, concurrent, real-life cognition.

In this study, we focus on spatial and temporal viewer re-orientation, and, to
this end, describe changes in depicted location and time for all cuts in the
motion picture ``Forrest Gump''. This movie is the core stimulus of the
\textit{studyforrest} project (\url{http://studyforrest}). Two fMRI datasets
are publicly available: 1) participants listening to an audio-movie version
\citep{HBI+14} and 2) a subset of the original participants watching the
audio-visual movie with simultaneous eye tracking \citep{HAK+16}. Additional
imaging data and movie annotations are available \citep{HDH+2015,LRS+2015},
including an individual localization of the \textit{parahippocampal place area}
\citep{SKG+16} that has been implicated in spatial perception and scene
processing \citep{EK1998}.

This new annotation extends the available knowledge about the structure of this
complex natural stimulus and enriches the overall studyforrest dataset. These
data can be used to investigate the formation of a representation of viewer
location and the perception of (speeded or negative) temporal progression in
the movie stimulus. For any study focusing on other aspects of real-life
cognition, these new data can serve as additional confound measures describing
key properties of major building blocks of this movie stimulus.


\section*{Materials and methods}
% Detailed account of the protocol used to generate the dataset

\subsection*{Stimulus}

% ~160 words till data legend
The annotated stimulus was a slightly shortened ($\approx$\unit[2]{h}) version
of the movie Forrest~Gump (R.~Zemeckis, Paramount Pictures, 1994) with dubbed
German soundtrack that is identical to the audio-visual movie annotated in
\cite{LRS+2015}. Further details on this particular movie cut, and how to
reproduce it from commercially available sources, are available in
\cite{HAK+16}.


\subsection*{Annotation procedure}

\begin{table*}[t]
\caption{Example annotations for four shots at the beginning of the movie. Note
that table headers do not literally correspond to column headers, see Data Legend
(ToD: time of day).}
\label{tab:example}
\begin{tabular}{lllllll}
\toprule
\textbf{time} & \textbf{major location} & \textbf{setting} & \textbf{locale} & \textbf{int/ext} & \textbf{flow of time} & \textbf{ToD}\\
\midrule
311.96  & Greenbow Alabama  & doctor's office  & doctor's office  & ext  & 0 & day\tabularnewline
318.28  & Greenbow Alabama  & main street  & in front of barbershop  & ext  & + & day\tabularnewline
343.04  & United States  & flashback countryside  & flashback countryside  & ext  & - & day\tabularnewline
353.08 & Greenbow Alabama  & main street  & in front of barbershop  & ext  & ++ & day\tabularnewline
\bottomrule
\end{tabular}
\end{table*}


First, the movie was explored by two people, one of whom has an academic
background in documentary film making, in order to generate a consistent list of
labels for depicted and recurring locations.

Subsequently, the actual annotation was performed by the first author using a
multi-pass strategy. The movie was manually inspected frame-by-frame to
determine the location of cuts (using the video editor Shotcut v16.02.01).
For each new shot (sequence between two cuts), a number of properties (described
below) were discerned and entered into a table. A total of four passes were
performed by the same observer in order to validate the annotation.

\subsection*{Data legend}

The annotation table contains one line per shot and seven columns: 1) a shot's
\textit{start time}, 2) a label for the shot's \textit{major location}, 3) a
label for the \textit{setting} within the location, 4) a label for the
\textit{locale} within the setting, 5) a flag indicating an \textit{interior or
exterior} setting, 6) a label for the type of \textit{temporal progression}
with respect to the previous shot, and 7) a label for the \textit{time of day}.
Further details are provided in the following sections. The respective column
header labels are given in parenthesis.


\subsubsection*{Shot start time (\texttt{time})}

A shot's start time is defined as the onset time of the first video frame of a
shot after a cut. Time stamps a provided in seconds of movie onset.

\subsubsection*{Location}

% TODO: This... doesn't sound accurate. Check?
Location was coded with three labels, each describing the depicted scenery
with an increasing level of detail.

\paragraph{Major location} (\texttt{major\_location}) provides a coarse
identification at the level of a town, county, or region where the respective
story is taking place. Examples are: ``Greenbow'' or ``Vietnam''.

\paragraph{Setting} (\texttt{setting}) further details the location by
distinguishing places at the same major location, but are not in direct sight of
each other. For example, Forrest Gump's elementary school and the high school's
football field are both in Greenbow, Alabama but are not part of the same
setting. A switch from one setting to another is typically synonymous with a
transition to a new scene in a cinematographic sense. If the camera switched
settings within a scene, the annotation deviates from the screenplay to make
explicit the switch to another setting.

\paragraph{Locale} (\texttt{locale}) subdivides settings into distinguishable
locales. Indoors, a locale is congruent with a particular room enclosed by
walls. For example, Forrest Gump's bedroom, the corridor downstairs, and the
corridor upstairs are three different rooms inside the Gumps' house (setting)
on the Gumps' property (major location). Outdoors, locales were distinguished
when they were separated by a logical boundary, substantial distance, or shared
no discernible landmarks. For example, the glade at the river and the location
of the wounded Bubba are two different locales in the embattled jungle (setting)
in Vietnam (major location). A locale's label is identical to its setting label
when only one locale is depicted for that setting.


\subsubsection*{Interior or exterior (\texttt{int\_or\_ext})}

This flag indicates whether a particular location is an open
(``\texttt{ext}'') or enclosed space (``\texttt{int}''), such as a building
or a vehicle.


\subsubsection*{Temporal progression (\texttt{flow\_of\_time})}

This label indicates the depicted progression of time between the previous and
the current shot. Four categories were distinguished: ``\texttt{-}'' labels a
flashback, or jump into the past, independent of the temporal distance;
``\texttt{0}'' indicates no noticeable break in the ongoing stream of time, for
example a sole change of viewing perspective; ``\texttt{+}'' represents
noticeable jumps in time, ranging from several seconds to about one or two
hours; and lastly ``\texttt{++}'' marks major time jumps from several hours
(e.g. night vs.~day) to several years.


\subsubsection*{Time of day (\texttt{time\_of\_day})}

This flag indicates whether a scene is at least partially illuminated by
sunlight. Consequently, daytime and twilight (early sunrises and late sun
settings) are labeled as ``\texttt{day}''. If sunlight is entirely missing, the
time of day is coded as ``\texttt{night}''.



\subsection*{Dataset content}

% ~30 words
The released annotation is a single, text-based, comma-separated-value (CSV)
formatted table.

The source code for all descriptive statistics and included in this paper is
available in \texttt{code/descriptive\_stats.py} (Python script).


\section*{Dataset validation}
% Information about any validation carried out and/or any limitations of the
% datasets, including any allowances made for controlling bias or unwanted
% sources of variability.

% ~170 words
To check for human error in the cut time annotation, timings were compared to
the results of an automatic detection algorithm and any deviation was manually
verified.

In summary, the shortened version of the movie comprises \NShots\ shots
(duration: min=\unit[\ShotLengthMin]{s}, max=\unit[\ShotLengthMax]{s},
median=\unit[\ShotLengthMedian]{s}, SD=\unit[\ShotLengthSD]{s}). There are
\NExteriorShots\ shots depicting outdoor locations and \NInteriorShots\
interior shots. Most shots take place during daytime (\NDayShots\ day
vs.~\NNightShots\ night). The majority of cuts involve no
noticeable discontinuities of depicted time (\NShotsTimeNoJump), but the are
\NShotsTimeSmallJump\ small and \NShotsTimeLargeJump\ large time jumps, as
well as \NShotsTimeFlashback\ flashbacks.

Table \ref{tab:stats} provides information on the portrayal of unique locations
in the movie.


\begin{table*}[b]
  \centering
  \caption{Descriptive statistics for all three levels of location annotation.
    \textit{Number of shots} indicates the total number of shots in the movie
    for any particular location. \textit{Number of consecutive shots} indicates
    how many shots are shown between two location changes at the respective
    level. \textit{Times revisited} indicates how often a location
    reappears in the movie after it was depicted for the first time.}
  \label{tab:stats}
  \begin{tabular}{rrrrrrrrrrrrr}
    %& \multicolumn{3}{l}{\textbf{Event definition min. agreement}} \\
    % header
    \toprule
    &
    \multicolumn{4}{c}{\textbf{major locations}} &
    \multicolumn{4}{c}{\textbf{settings}} &
    \multicolumn{4}{c}{\textbf{locales}} \\
    \midrule
    number of unique & \multicolumn{4}{c}{\NMajorLocations} & \multicolumn{4}{c}{\NSettings} & \multicolumn{4}{c}{\NLocales} \\
    \midrule
    % subheader
    & min & med. & mean & max &
      min & med. & mean & max &
      min & med. & mean & max \\\\
    number of shots
    & \ShotsPerMajorLocationMin & \ShotsPerMajorLocationMedian & \ShotsPerMajorLocationMean & \ShotsPerMajorLocationMax 
    & \ShotsPerSettingMin & \ShotsPerSettingMedian & \ShotsPerSettingMean & \ShotsPerSettingMax 
    & \ShotsPerLocaleMin & \ShotsPerLocaleMedian & \ShotsPerLocaleMean & \ShotsPerLocaleMax
    \\
    number of consecutive shots
    & \ConsecShotsPerMajorLocationMin & \ConsecShotsPerMajorLocationMedian & \ConsecShotsPerMajorLocationMean & \ConsecShotsPerMajorLocationMax 
    & \ConsecShotsPerSettingMin & \ConsecShotsPerSettingMedian & \ConsecShotsPerSettingMean & \ConsecShotsPerSettingMax 
    & \ConsecShotsPerLocaleMin & \ConsecShotsPerLocaleMedian & \ConsecShotsPerLocaleMean & \ConsecShotsPerLocaleMax
    \\
    times revisited
    & \NTimesMajorLocationsRevisitedMin & \NTimesMajorLocationsRevisitedMedian & \NTimesMajorLocationsRevisitedMean & \NTimesMajorLocationsRevisitedMax 
    & \NTimesSettingsRevisitedMin & \NTimesSettingsRevisitedMedian & \NTimesSettingsRevisitedMean & \NTimesSettingsRevisitedMax 
    & \NTimesLocalesRevisitedMin & \NTimesLocalesRevisitedMedian & \NTimesLocalesRevisitedMean & \NTimesLocalesRevisitedMax
    \\
    \bottomrule
  \end{tabular}
\end{table*}


\section*{Data availability}

\texttt{This section will be auto-generated.}


\section*{Author contributions}
%In order to give appropriate credit to each author of an article, the
%individual contributions of each author to the manuscript should be detailed
%in this section. We recommend using author initials and then stating briefly
%how they contributed.
CH designed, performed, and validated the annotation, and wrote the manuscript.
MH provided critical feedback on the procedure and wrote the manuscript.

\section*{Competing Interests}
No competing interests were disclosed.

\section*{Grant Information}

Michael Hanke was supported by funds from the German federal state of
Saxony-Anhalt and the European Regional Development Fund (ERDF), Project: ,
Project: Center for Behavioral Brain Sciences.

\section*{Acknowledgements}
%This section should acknowledge anyone who contributed to the research or the
%article but who does not qualify as an author based on the criteria provided
%earlier (e.g. someone or an organisation that provided writing assistance).
%Please state how they contributed; authors should obtain permission to
%acknowledge from all those mentioned in the Acknowledgements section.  Please
%do not list grant funding in this section (this should be included in the
%Grant information section - See above).
We are grateful to Daniel Kottke for cross-checking the timing of the cuts in
the movie using an automated detection routine, and Alex Waite for editing this
manuscript. We would also like to thank Gavin Theren for patiently sharing his
cinematographic knowledge during movie watching and for his high-level
gastronomic skills.

%\nocite{*}
{\small\bibliographystyle{unsrt}
\bibliography{references,bibliography_forrest_gump}}

\end{multicols}
\end{document}

% vim: textwidth=80 colorcolumn=81
