\documentclass[10pt,a4paper,twocolumn]{article}
\usepackage{endfloat}
\usepackage{f1000_styles}
\usepackage{listings}
\usepackage{units}
\usepackage[colorlinks]{hyperref}
\usepackage{url}
\usepackage{appendix}
\usepackage{soul}

\definecolor{dkgreen}{rgb}{0,0.6,0}
\definecolor{gray}{rgb}{0.5,0.5,0.5}
\definecolor{mauve}{rgb}{0.58,0,0.82}

\begin{document}

\newcommand{\NShots}{870}


\title{An annotation of locations visited in the motion picture ``Forrest Gump''}

\author[1]{Christian Häusler}
\author[1,2]{Michael Hanke}

\affil[1]{Psychoinformatics lab, Department of Psychology, University of Magdeburg, Universit\"{a}tsplatz 2, 39106 Magdeburg, Germany}
\affil[2]{Center for Behavioral Brain Sciences, Magdeburg, Germany}
\maketitle
\thispagestyle{fancy}

\begin{abstract}
% Abstracts should be up to 300 words and provide a succinct summary of the
% article. Although the abstract should explain why the article might be
% interesting, care should be taken not to inappropriately over-emphasise the
% importance of the work described in the article. Citations should not be used
% in the abstract, and the use of abbreviations should be minimized.

\todo[inline]{Write an abstract}

\end{abstract}

\listoftodos[ToDo notes]

\clearpage

%The format of the main body of the article is flexible: it should be concise
%and in the format most appropriate to displaying the content of the article.

\section*{Introduction}
% Rationale for creating the dataset(s) and/or objectives for the experiment
% resulting in the dataset - why the data were gathered or produced.

\todo[inline]{Why was it useful to make this annotation? What kind of cognitive processing
can we investigate with the movie dataset now that we have this annotation?
How is this kind of research normally done (without movie datasets)? Link to
cinematography. Cite key papers.}

\section*{Materials and methods}
% Detailed account of the protocol used to generate the dataset

\subsection*{Stimulus}

The annotated stimulus was a slightly shortened ($\approx$\unit[2]{h}) version
of the movie ``Forrest~Gump'' (R.~Zemeckis, Paramount Pictures, 1994) with
dubbed German soundtrack, and is identical to the audio-visual movie annotated
in \cite{LRS+2015} with respect to portrayed emotions. Further details on this
particular movie cut, and how to reproduce it from commercially available
sources are available in \cite{HAK+16}.


\subsection*{Annotation procedure}
\begin{itemize}
\item Beispiel der Annotation: 
\begin{table}[h]
\begin{tabular}{lllllll}
\multicolumn{7}{l}{Table 1. Fucking Annotation}\tabularnewline
\hline 
time & major location & setting & locale & int or ext & flow of time & time of day\tabularnewline
00:05:00:14  & Greenbow Alabama  & doctor's office  & doctor's office  & ext  & 0 & day\tabularnewline
00:05:11:24  & Greenbow Alabama  & main street  & in front of barbershop  & ext  & + & day\tabularnewline
00:05:18:07  & United States  & flashback countryside  & flashback countryside  & ext  & - & day\tabularnewline
00:05:43:01 & Greenbow Alabama  & main street  & in front of barbershop  & ext  & ++ & day\tabularnewline
\hline 
\end{tabular}
\end{table}

\item figure with examples for the four conditions early in the movie (and
additional anno, e.g. +, int, day):
\end{itemize}
The shortened movie was annotated manually using the free, open source,
cross-platform video editor Shotcut (version 16.02.01). It offers
the possibility to watch a movie frame by frame and read a frame's timing
with the precision of the movie frame rate. A separate spreadsheet
software (LibreOffice Calc, version 5.1.4.2) was used to write the
actual annotation into a text file with comma-separated values (structure.csv). 

\subsection*{Annotation content}

The annotation file comprises seven columns: One column for 1) the
shot's start time, 2) the shot's major location, 3) it's setting,
4) it's locale, 5) an interior exterior distinction of the location,
6) it's temporal order in respect to the previous shot and 7) it's
time of day. 

\subsubsection*{Shot's start time}

A shot's start time is defnied as the first frame of a shot after
a cut. The timings are written in the format hour:minute:second:frame.
The values for the frame range from 0 to 24, reflecting the movie's
frame rate of 25 frames per second. 

To check for human errors and validate the timings, the manual annotation was
compared to an annotation created by an in-house developed, automated cut
detection algorithm {[}wer hat's gemacht; ggf. wo Unterschiede waren{]}. In
summary the shortened version of the movie comprises \NShots\ shots (duration:
min=\unit[\ShotLengthMin]{s}, max=\unit[\ShotLengthMax]{s},
median=\unit[\ShotLengthMedian]{s}, SD=\unit[\ShotLengthSD]{s}).


\subsubsection*{Annotation of locations}

The location shown at the beginning of every shot was written into
three different columns, reflecting increasing granularity: From the
major location via setting through to locale. The movie's screenplay
by Robert Roth (first draft; December 18, 1992) served as a template
for the naming of the filmed locations. Where reasonable namings were
adjusted. Further spatial information was added to fit into the three
columns.

Firstly, the major location represents the location on a more abstract
level. It represents the city resp. town or, if the exact city is
unknown, the region or country. For example Greenbow, Washington D.C.
or Vietnam. 

Secondly, the setting represents a subcategory for places which are
in the same major location, but are not in direct sight. For example
Forrest Gump's elementary school and the football field, which are
both in the same major location of Greenbow, Alabama. A switch from
one setting to another setting is mostly, but not always, congruent
with a switch to another scene in a cinematographic sense.

Lastly, a setting was further subdivided into distinguishable locales.
Indoors, a locale is congruent with a particular room bounded by its
walls or floor. For example Forrest Gump's bedroom, the corridor downstairs
and the corridor upstairs are three different rooms inside the Gumps'
house (setting) on the Gumps' property (major location). Outdoors,
different locales were assumed if two places could be considered as
different rooms due to their distance to each other, due to logical
borders or due to no or only a small number of overlapping objects
and geometrical features {[}Beispiele{]}. A locale is equal to it's
setting, when a setting depics only one locale. 


\subsubsection*{Interior exterior distinction }

Regarding the interior exterior distinction, a location inside a building
or inside a car was labeled with ``int'', while locations outdoors
were labeled with ``ext''. 


\subsubsection*{Temporal order}

Firs, every flashback, no matter how short the skipped time is, was
coded as >>-<<. Second, a >>0<< was chosen, when the cut is supposed
to show no noticable break in the ongoing stream of time. In other
words, in those cases the action is supposed to be cutted in a continouos
fashion, so the action is only shown from a different viewpoint after
the cut. Third, a ``+'' represents noticeable breaks in the ongoing
stream of time, ranging from mostly several seconds to about one to
two hours. Finally, a ``++'' is used for obvious, major time jumps,
ranging from mostly several hours (e.g. day to night) to a couple
of years.


\subsubsection*{Time of day }

The last column represents the time of day as a binary variable. If
the sun light is at least partially illuminating the scene, the time
of day is coded as ``day''. By this rationale twilights (early sunrises
and late sun settings) are also subsumed as ``day''. If sun light
is entirely missing, the time of day is coded as ``night''. 


\subsubsection*{from raw annotation to experimental conditions}

\todo[inline]{this section could be stripped of all FSL/MRI references
  and converted into a ``validation'' that lists possible conditions
  and counts the respective events in each, so people get a sense of
  what is there. I would also remove the ``no-cut'' references, as those
  are relatively arbitrary decision that are only meaningful in a particular
context, not for the general annotation data}

The data in the raw annotation were converted to event onset files
for later use in FSL (ev3 format) by means of two python scripts. 

The first script (01\_structure\_annot\_to\_all\_timings.py) took
the annotation as input and created a new csv-file (structure\_all\_timings\_10sec.csv)
with columns ``time\_stamp'', ``seconds'' and ``condition''.
The time stamp was a copy of the time stamp given in the raw annotation.
It served for later visual inspection of conditions' correctness by
easily copy and pasting it into Shotcut. 

The column ``seconds'' coded the timing of an event beginning from
the movie's start in the format seconds.milliseconds. The column ``condition''
coded an event's respective condition (``no\_cut'', ``perspective\_change'',
``locale\_change'', ``setting\_change''). The rules to produce
the condition from the annotation were as follows: When the major
location, the setting and the room in one line of the annotation were
the same as in the previous line, the script returned ``perspective\_change''.
When only the locale was different to the previous line line, the
script returned ``locale\_change''. When the setting or the major
location was / were different, the script returned ``setting\_change''. 

Additionally, the timings of the control condition (``no\_cut'')
were fit{[}ted{]} into longer lasting shots by the following rationale:
no\_cuts had to be apart from other no\_cuts and apart from any cut
by ten seconds {[}better: separated from?{]}. Thus, a shot's duration
was divided by a factor of ten seconds. The biggest integral multiple
of ten yielded the number of possible time points for events of the
no cut condition. Then, the shot's duration was divided by it's corresponding
integral multiple to yield the temporal distance of the events in
every shot. By this procedure within individual shots, time points
are temporally equidistant apart to / from each other. Furthermore,
this yielded a slight jitter interval past / after / follwing the
ten second mark different for every shot. An additional jitter, different
for every no cut was drawn from a gaussian distribution with mean
= 0 and standard deviation of 1500 ms. The computational results {[}given
in milliseconds{]} were then aligned to the timing of the movie's
nearest frame. Finally, timings for all four conditions were written
to file. 

The total number of events for the conditions were: 386 events for
perspective\_new (occuring in shots with median shot length of 3.96
s; SD = 7.484 s), 208 events for perspective\_old (Mdn = 3.5 s; SD
= 7.470 s), 96 events for setting\_new (Mdn = 8.92; SD = 21.882 s),
90 events for setting\_old (Mdn = 7.68; SD = 9.440 s), 89 events for
locale\_change (Mdn = 6.96; SD = 8855 s), 148 events for no\_cut (with
the chosen factor of ten seconds; in 202 different shots; Mdn = 16.96
s; SD = 16.090s)\todo{stats in this paragraph need to be computed from
annotation}


\subsection*{Dataset content}

\todo[inline]{What exactly is released, in what format?}

\paragraph{Source code}

The full source code for all descriptive statistics and figures included in
this paper is available in \texttt{descriptive\_stats.py} (Python script).


\section*{Dataset validation}
% Information about any validation carried out and/or any limitations of the
% datasets, including any allowances made for controlling bias or unwanted
% sources of variability.

\todo[inline]{Place any information and illustrative figures here that help
people figure out whether these annotation are useful/trustworthy for them}

\section*{Data availability}

\texttt{This section will be auto-generated.}


\section*{Author contributions}
%In order to give appropriate credit to each author of an article, the
%individual contributions of each author to the manuscript should be detailed
%in this section. We recommend using author initials and then stating briefly
%how they contributed.
CH design, performed, and validated the annotation, and wrote the manuscript.
MH provided critical feedback on the procedure and wrote the manuscript.

\section*{Competing Interests}
No competing interests were disclosed.

\section*{Grant Information}

Michael Hanke was supported by funds from the German federal state of
Saxony-Anhalt and the European Regional Development Fund (ERDF), Project: ,
Project: Center for Behavioral Brain Sciences.

\section*{Acknowledgements}
%This section should acknowledge anyone who contributed to the research or the
%article but who does not qualify as an author based on the criteria provided
%earlier (e.g. someone or an organisation that provided writing assistance).
%Please state how they contributed; authors should obtain permission to
%acknowledge from all those mentioned in the Acknowledgements section.  Please
%do not list grant funding in this section (this should be included in the
%Grant information section - See above).
We are grateful to Daniel Kottke for cross-checking the timing of the cuts in
the movie using an automated detection routine.

%\nocite{*}
{\small\bibliographystyle{unsrt}
\bibliography{references}}

\end{document}

% vim: textwidth=80 colorcolumn=81
